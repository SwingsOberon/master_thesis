\documentclass{report}

\raggedright

\begin{document}

\chapter{Method}

The main goal of this work is to achieve a secure open platform on the hardware.

\section{Detailed Problem}

\begin{itemize}
\item Lots of functionality \begin{itemize}
\item Sensitive data
\item Mobile computing
\end{itemize}
\item Performance driven \begin{itemize}
\item All resources to functionality
\item Little room for security
\item Extensive security measures needed
\end{itemize}
\item Hardware security solution \begin{itemize}
\item Security embedded in hardware
\item Better for performance
\item Correct implementation required
\end{itemize}
\end{itemize}

\section{System Model}

\begin{itemize}
\item Open platform \begin{itemize}
\item Multiple software providers
\item Platform owned by user
\end{itemize}
\item Secure software execution \begin{itemize}
\item Software isolation
\item Secure data storage
\end{itemize}
\end{itemize}

The system model describes an open platform with no or minimal trust among stakeholders.

\section{Attacker Model}

\begin{itemize}
\item Physical access \begin{itemize}
\item Threats
\item Vulnerabilities
\end{itemize}
\item OS/Firmware attacks \begin{itemize}
\item Threats
\item Vulnerabilities
\end{itemize}
\item Software attacks \begin{itemize}
\item Threats
\item Vulnerabilities
\end{itemize}
\end{itemize}

The attacker has physical access, can launch OS/firmware and software attacks. The Trusted Platform Module is assumed to be tamper resistent.

\section{Solution}

\begin{itemize}
\item Secure boot \begin{itemize}
\item Root of Trust
\item Chain of Trust
\item Secure starting point
\end{itemize}
\item User attestation \begin{itemize}
\item Integrity (control flow, data structures, ...)
\item Authenticity (code, ...)
\end{itemize}
\item Trust \begin{itemize}
\item Execution
\item Data protection
\end{itemize}
\end{itemize}

Ideally the device is started with secure boot, this makes sure the SW is started from a known secure state. 
\medskip

During operation the user should be able to attest whether their device is still in a secure state.
\medskip

This can be done using a TA that makes measurements on their device and reports back to them.
\medskip

These measurements are checking the integrity of the code section of the running applications and OS.

\end{document}