\documentclass{report}

\raggedright

\begin{document}

\chapter{Method}

The main goal of this work is to achieve a secure open platform on the hardware.

\section{System Model}

The system model describes an open platform with no or minimal trust among stakeholders.

\section{Attacker Model}

The attacker has physical access, can launch OS/firmware and software attacks. The Trusted Platform Module is assumed to be tamper resistent.

\section{Booting Process}

The secure boot makes sure that the device starts in a secure, trusted and known state.

\section{OP-TEE integration}

OP-TEE is the TEE framework used in this thesis and is integrated with the linux distribution that is booted on the hardware.

\section{Secure Applications}

Secure applications make use of the TEE capabilities of ARM TrustZone with the help of OP-TEE. These secure applications make use of the secure world execution for sensitive tasks.

\section{Security Properties}

The secure boot process makes sure that TrustZone works as intended which should give confidence in the belief that secure execution of secure applications is guaranteed.

\end{document}