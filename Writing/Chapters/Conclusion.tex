\chapter{Conclusion}

\paragraph*{}
The goal of this thesis is to come up with security measures that can protect an open platform. The use case of a smartphone is specifically looked at, because currently the large companies that produce these devices, disrupt the openness of the system while implementing security mechanisms. Thanks to the introduction of the PinePhone, a hardware platform with the necessary security features and openness has become available. Using this hardware, it is reviewed what RoT could be used. This proves to be a sensitivity point towards the openness of the system. Furthermore, attestation is investigated as a security system for the platform, taking into account the user expectations. Finally, the achieved security guarantees are evaluated to support our claims. Reviewing the RoT is attempted by implementing secure boot on the PinePhone, but turned out to be more challenging than expected. Therefore this is pushed back to the discussion. There the nuances of what needs to be taken into account to keep the system open, are elaborated. The attestation on the other hand, is implemented based on existing work: the solution is modified to fit better into the use case. Evaluating the user expectations is touched on during the experiments by giving a performance and security analysis. The achieved security is compared with related solutions that do not take into account the openness of the system. The implementation of the attestation focuses on measuring the executable memory pages that are present in RAM and in use by running processes. The details of this process are elaborately explained in the implementation. The experiments show that although the measuring and thus checking the integrity works, it performs very slowly compared to the work on which this implementation is based. More effort needs to be put into optimizing the code to achieve results that are feasible. Based on the comparison with closed system, it can be stated that the proposed solution provides similar security guarantees, given that the RoT was instantiated in a secure manner.