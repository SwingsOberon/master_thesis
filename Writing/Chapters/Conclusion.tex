\documentclass{report}

\raggedright

\begin{document}

\chapter{Conclusion}

To achieve secure execution on the PinePhone some requirements need to be met. One of these requirements is that a chain of trust is achieved which is done using secure boot in this case.

\section{Related work}

\begin{itemize}
\item Secure boot, Trusted boot and remote attestation for ARM TrustZone-based IoT Nodes
\item DAA-TZ: An Efficient DAA Scheme for Mobile Devices Using ARM TrustZone
\item SecTEE: A Software-based Approach to Secure Enclave Architecture Using TEE
\item TZ-MRAS: A Remote Attestation Scheme for the Mobile Terminal Based on ARM TrustZone
\item TrustShadow: Secure Execution of Unmodified Applications with ARM TrustZone
\end{itemize}

\section{Comparison of Approaches}

\begin{itemize}
\item Effectiveness \begin{itemize}
\item Reached goal
\item Defends most variety of attacks
\item Stronges security guarantees
\end{itemize}
\item Assumptions \begin{itemize}
\item Least assumptions
\item Most realistic
\end{itemize}
\end{itemize}

\section{Future Improvements}

\begin{itemize}
\item Best approach \begin{itemize}
\item Overview
\item Reasoning
\end{itemize}
\item Weaknesses
\item Possible improvements \begin{itemize}
\item inspiration from Lightweight and Flexible Trust Assessment Modules for the Internet of Things
\end{itemize}
\item Different solutions
\end{itemize}

To allow the user to attest their device it is important that Trusted I/O is used to inform the user about the outcome of the attestation process.
\medskip

The attestation application can be seen as one module that can be accompanied with a variety of different modules to increase the amount of checks that can be executed to check more possible attacks/ vulnerabilities.

\end{document}