\chapter{Introduction}

\paragraph*{} 
Smartphones have become an essential part of our daily lives. This can be seen when comparing the number of smartphone users \cite{smartphones} and the worldwide population \cite{people}, because in 2020 80\% of the entire population owned a smartphone device. These devices can be spotted everywhere: people use them at home, on the bus or even at work. In most cases, these phones are only occasionally used for text messaging or calling but very often for reading e-mails, surfing on the web or even for services like e-banking. Because of this wide range of functionalities, some people may even replace their personal computer with a smartphone entirely. The success of smartphones lies in their ease of use and always being accessible: people just carry it in their pockets. Besides the user having access to his phone all the time, a smartphone also has or can have access to the internet all the time. The internet is the gate to lots of services which are often accessed through mobile devices \cite{internet}, in 2021 over 90\% of internet users accessed the internet through their smartphone.

\paragraph*{}
Devices making lots of connections on the go while also utilizing online services, for which sensitive data is required may introduce security vulnerabilities. The fact that people are using their smartphones for services like online banking, or even consulting health-related reports implies that some sensitive data must be stored on these devices, or at least be present while they are interacting with it. While this data should be protected very well, it is present on an Internet of Things (IoT) device for which lots of security challenges exist \cite{LitoussiMohamed2020Isca}. The hardware similarities between smartphones and IoT devices are more prominent than one might expect. Because the architecture of smartphones is often built around the same or very similar System on Chip (SoC) as IoT devices. The main issue here is that IoT devices are designed for performance: they only have a small number of tasks but these need to be executed as fast or as energy efficient as possible. This also applies to smartphones because while the functionality of a smartphone is close to that of a personal computer, the hardware is not. This weak link in the hardware gives rise to multiple different attack strategies that adversaries can utilize to steal sensitive data from smartphone users. 

\paragraph*{} 
Recently improvements have been made in the area of IoT and smartphone hardware by extending the processors of these devices with features that make it possible to set up a Trusted Execution Environment (TEE). A TEE can increase the security of an IoT device, which is often achieved by utilizing core security services for critical operations. Examples of these critical operations are cryptographic operations, storing data in secure memory, or accessing Input and Output (I/O) through trusted paths. The majority of smartphones use SoC designed by ARM, very often equipped with ARM TrustZone capabilities. Large smartphone manufacturers like Samsung utilize these features to build a security solution on them, Samsung KNOX \cite{KNOX} for instance. The downside to these solutions is the disadvantage that the manufacturer stays in control of the smartphone even after it has been sold. He decides which software is allowed to run on the device and which is not. Recently the PinePhone \cite{PinePhone} was introduced, which is an open smartphone platform that also has an ARM SoC with TrustZone enabled. On top of the openness of the platform itself, Linux is the rich Operating System (OS) that runs on this device and the recommended kernel for the Secure World (SW) is OP-TEE \cite{OP-TEE}. These are both large open source projects with an active community. 

\subsection*{Contributions}

\paragraph*{}
We attempt to increase the security of a smartphone without closing off the system with respect to the user or third-party software providers. This means that the user should be in control and be able to decide what software can run on his device and what not, instead of the manufacturer of the device. This is achieved by taking integrity measurements of the processes running in the user space and rich OS environment. The measuring process runs inside the TEE to provide strong security guarantees for its execution. This method is based on existing work \cite{LingZhen2021Sbtb}, which focuses on IoT nodes and the performance of the attestation. The focus in this thesis is shifted towards making the solution work on a PinePhone while respecting the openness of the device. To allow for a direct comparison between the performance achieved in the paper and our implementation, some experiments in the paper are redone. To allow others to easily reproduce or review the work that has been done, all code is made available in open source at \url{https://github.com/SwingsOberon/master_thesis}. Of course, the performance is only a small aspect of the analysis of the solution. To make the security analysis, this solution is compared to similar work. This comparison discusses what solution is the most effective, and what solution achieves the most in terms of defended attacks while keeping in mind the reality and feasibility of the assumptions that are made. Finally, it is evaluated which type of solution is the most promising as the direction for future work based on the comparison with similar alternatives.

\paragraph*{}
Based on the performed work, executed experiments and discussed comparisons the following questions are attempted to be answered: \begin{itemize}
\item What can be used as a Root of Trust (RoT) in such an open system?
\item Can attestation be used to increase the security of an open platform without disrupting the openness of the system?
\item Does this security solution meet the expectations of all smartphone stakeholders?
\item How does this type of security solution compare with the ones in closed systems?
\end{itemize}

\subsection*{Outline}

\paragraph*{}
In the next chapter, more background information about, among other things, Remote Attestation (RA), and ARM TrustZone is given. In the third chapter, the methods to solve the problem are explained. In chapters 4 and 5, the implementation of the integrity measurement program is elaborated upon and the outcome of the experiments are made clear respectively. The sixth chapter includes a discussion about this thesis informing the reader about related work and future work. The final chapter concludes the presented work.
