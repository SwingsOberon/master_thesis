\documentclass{report}

\raggedright

\begin{document}

\chapter{Introduction}

\begin{itemize}
\item Smartphones everywhere \begin{itemize}
\item Everyone has one
\item Interchangeable with PC
\end{itemize}
\item Sensitive data \begin{itemize}
\item Lots of traffic
\item Personal data stored
\end{itemize}
\item Comparable to IoT \begin{itemize}
\item Hardware similarities
\item Security features (lot less than PC)
\end{itemize}
\end{itemize}

Smartphones are everywhere, being used for more and more sensitive data. Hacking into these devices should be made as hard as hacking into someone's personal computer because for many those two have become interchangeable.

\section{Problem statement}

\begin{itemize}
\item IoT security \begin{itemize}
\item Minimize overhead
\end{itemize}
\item PC functionality \begin{itemize}
\item Banking, e-Health and mails
\end{itemize}
\item Missmatch \begin{itemize}
\item Sensitive data requires good security
\item Functionality is pushed but security lags behind
\end{itemize}
\end{itemize}

The hardware in smartphones is comparable to that of IoT devices, it is more powerfull in many occasions but the design principles are often the same. The problem with this is that IoT devices are not very secure, smartphones are in that sense lagging behind on security compared to how they are used (banking, health and identification applications).

\section{Contributions}

\begin{itemize}
\item Reproduction of paper \begin{itemize}
\item Replicated solution as closely as possible
\item Evaluation about results compared to original
\end{itemize}
\item Open source code (proof of concept) \begin{itemize}
\item Enable easier reproduction/verification in the future
\end{itemize}
\item Extra experiment measurements \begin{itemize}
\item Comparable experiments (to be able to compare)
\item More elaborate experiments (to allow better decision making)
\end{itemize}
\item Comparison with similar solutions \begin{itemize}
\item Overview of comparable papers (pros and cons)
\item Weak points and strenghts of the reproduced paper compared to the others
\end{itemize}
\end{itemize}

Major producers of smartphone chips are adding hardware support for security (Intel SGX, ARM TrustZone). The focus of this thesis lies in using ARM TrustZone to achieve a secure open platform from a smartphone equiped with ARM TrustZone.

\section{Outline}

In the next chapter more background information about among other things ARM TrustZone and Remote Attestation will be given. In the third chapter the methods secure applications will be explained. In the fourth chapter the goal and outcome of the experiments will be made clear. The final chapter will conclude this thesis informing the reader about limitations of this work and possible future directions of research.

\end{document}