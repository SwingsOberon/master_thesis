\documentclass{report}

\raggedright

\begin{document}

\chapter{Implementation}

\section{Attestation TA}

\subsection*{Trusted application}

\paragraph*{Hashing}
the memory pages of the text section of a program is the very first step the TA does when attesting a program.

\paragraph*{Storing reference values}
needs to be done when the device is in a known secure state, the memory pages will be hashed and these hashes are stored for later comparison.

\paragraph*{Comparing}
the hash of a memory page with its reference value is the actual attesting step, from this comparison it should be clear whether the integrity of the memory page has been violated.

\paragraph*{Notifying}
the user is the final step to inform them about the problem to allow them to take action, this could be rebooting to ensure a secure known state again or ask for help from a specialist.

\subsection*{NW OS dependencies}

\paragraph*{Retrieving address}
needs to be done from within the NW OS at the moment, this is because the datastructures that contain this information are owned by the NW OS.

\paragraph*{Translating address}
is another functionality for which the rich OS is used, this is also due to the fact that these translations are easily determined from the NW OS datastructures.

\subsection*{Extensions}

\paragraph*{Trusted IO}
could be used to inform the user of the problem (which program has been tampered with for instance), it could also take on a more coarse grained form that an led licht signals the user that some piece of software has failed the attestation which is easier but less usefull.

\paragraph*{Becoming independent}
from the rich OS in the normal world seems like a very important step because otherwise OS/Firmware attacks are still a threat.

\paragraph*{Detailed attestation}
is necessary, lots of software attacks are based on the used datastructures and don't impact the text section of code of programs.

\end{document}