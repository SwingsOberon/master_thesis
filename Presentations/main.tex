\documentclass[]{beamer}
\usetheme{KUL}
\usepackage{multirow}
\usepackage{multicol}
\usepackage{tikz}
\usepackage{ulem}
\usepackage{todonotes}
\usepackage{siunitx}
\newcommand\itemS{\item[\textbf{\S}]}
\definecolor{darkgreen}{rgb}{0,0.598,0.199}
\usepackage{times} % set font on times new roman
\usepackage{eurosym} % package for Euro sign
\usepackage{lineno}   % package for line numbering
\usepackage{hyperref} % this is for url links
\usepackage{subcaption}  % this package enables one to put several figures next to each other
\usepackage{textcomp}
\usepackage{setspace}
\usepackage{gensymb}
\usepackage{tikz}
\usetikzlibrary{positioning}
\usetikzlibrary{matrix, arrows, graphs}
\usetikzlibrary{backgrounds}
\usetikzlibrary{calc}
\usepackage{amsmath}
\usepackage{adjustbox}


\title{Thesis}
\subtitle{Building A Secure and Open \\ IoT Platform with ARM TrustZone}
\author{Oberon Swings}
\institute{KU Leuven}
\date{\today}

\newenvironment{code}{\fontfamily{ccr}\selectfont}{\par}

\begin{document}

{
		\setbeamertemplate{headline}{} %define local, empty header for title page
		\setbeamertemplate{footline}{ref image: https://developer.arm.com/ip-products/security-ip/trustzone} %define local, empty footer for title page
		\maketitle
	}
	\addtocounter{framenumber}{-1} % We don't count the title page
	
\iftrue
% Table of Contents
\begin{frame}{Outline}
	\hfill	{\large \parbox{.961\textwidth}{\tableofcontents[hideothersubsections]}}
\end{frame}
\fi

\section{Secure Open Platform}

\begin{frame}{Goals}
Goals of an open platform
\end{frame}

\begin{frame}{Problems}
Security is hard to guarantee in this setting
\end{frame}

\begin{frame}{Security}
Security goals of an open platform
\end{frame}

\section{ARM TrustZone}

\begin{frame}{Trusted Execution Environment}
What is a Trusted Execution Environment, difference between SEE and TEE
\end{frame}

\begin{frame}{Secure and normal world}
How the hardware enforces security
\end{frame}

\begin{frame}{Root of Trust}
Root of trust is needed to achieve these goals
\end{frame}

\section{PinePhone}

\begin{frame}{Hardware}
Available hardware and support
\end{frame}

\begin{frame}{Application}
Open platform for mobile computing
\end{frame}

\begin{frame}{OP-TEE}
Open Portable Trusted Execution Environment on PinePhone
\end{frame}

\section{Research}

\begin{frame}{Research Question(s)}
Can the PinePhone be turned into a secure open IoT platform?
\begin{itemize}
\item What ARM TrustZone features does OP-TEE make availablee when being ported onto a PinePhone?
\item Is it feasible to secure boot the PinePhone and in this way achieve a root of trust?
\item Can the I/O of the PinePhone be secured using OP-TEE and ARM TrustZone?
\end{itemize}
\end{frame}

\begin{frame}{Hypothesis}
OP-TEE can be ported onto a PinePhone and will atleast enable secure boot and secure I/O.
Booting process will be slowed down but not to an unpleasant extent.
I/O will be slower due to switching between worlds, but I/O always suffers from OS overhead so the added overhead should be minimal.
\end{frame}

\section{Progress}

\begin{frame}{Past}
Qemu emulator on laptop to play around with OP-TEE and secure applications.
\end{frame}

\begin{frame}{Present}
Booting the PinePhone with OP-TEE
\end{frame}

\begin{frame}{Future}
Tweaking the booting process to use secure boot
Writing secure application to make use of secure I/O,...
\end{frame}


\end{document}