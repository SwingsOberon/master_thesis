\documentclass{article}

\raggedright

\begin{document}

\section{SEDA: Scalable Embedded Device Attestation}

\subsection{Introduction}

\subsubsection{Overview}

We design SEDA, Scalable Em-
bedded Device Attestation, which is, to the best of our knowledge,
the first attestation scheme for large-scale swarms. SEDA repre-
sents the first step in a new line of research on multi-device attesta-
tion. Although SEDA adheres to the common assumption – made
in most (single-prover) attestation techniques – of ruling out phys-
ical attacks on devices, we discuss mitigation techniques for such
attacks in Section 9.

\subsubsection{Contributions}

\begin{itemize}
\item First Swarm Attestation Scheme
\item Security Model \& Analysis
\item Two Working Prototypes
\item Performance Analysis
\end{itemize}

\subsection{Swarm Attestation}

\subsubsection{Requirements}

\begin{itemize}
\item Support the ability to remotely verify integrity of the swarm (S) as a whole.
\item Be more efficient than individually attesting each device (D) in S.
\item Not require the verifier (VRF) to know the detailed configuration of S.
\item Support multiple parallel or overlapping attestation protocol instances.
\item Be independent of the underlying integrity measurement mechanism used by devices in S.
\end{itemize}

\subsubsection{Adversary Model}

As common in the attestation literature [16,
24, 47, 48] we consider software-only attacks. This means that,
although the adversary, denoted as ADV, can manipulate the soft-
ware of (i.e., compromise) any device D in S, it cannot physically
tamper with any device. However, ADV can eavesdrop on, and ma-
nipulate, all messages between devices, as well as between devices
and VRF. Furthermore, we rule out denial-of-service (DoS) at-
tacks since ADV typically aims to remain stealthy and undetected
while falsifying the attestation result for VRF.

\subsubsection{Protocol Description}

SEDA has two phases: (1) an off-line phase whereby devices
are introduced into the swarm, and (2) an on-line phase performing
actual attestation. The off-line phase is executed only once and
consists of device initialization and device registration. The on-line
phase is executed repeatedly for every attestation request from a
verifier VRF.

\subsubsection{Swarm Attestation}

VRF starts attestation of S by sending an
attestation request attest (containing a random challenge) to D 1 .
VRF can randomly chose any device in S as D 1 or depending on
its location or preference. Recall that VRF might be remote, or
within direct communication range of one or more swarm devices.
Eventually, VRF receives an attestation report from D 1 . VRF
outputs a bit b = 1 indicating that attestation of S was successful,
or b = 0 otherwise. VRF starts
the protocol by sending a nonce N to D 1 . It, in turn, generates a
new q and runs attdev with all its neighbors, which recursively run
attdev with their neighbors. Note that N prevents replay attacks on
communication between VRF and D 1 while the purpose of q is to
identify the protocol instance and to build the spanning tree. Even-
tually, D 1 receives the accumulated attestation reports of all other
devices in S.


\end{document}
