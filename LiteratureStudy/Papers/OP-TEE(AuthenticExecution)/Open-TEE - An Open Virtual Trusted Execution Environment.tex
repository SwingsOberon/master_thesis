\documentclass{article}

\raggedright

\begin{document}

\section{Open-TEE - An Open Virtual Trusted Execution Environment}

\subsection{Introduction}

The following contributions are made: \begin{itemize} 
\item Design and implementation of a virtual TEE called Open-TEE which conforms to GlobalPlatform Specifications.
\item It is shown that Open-TEE is efficient, hardware-independent and allows a developer to carry out much of the development life cycle of standard-compliant TEE applications using popular application development environments they already use.
\end{itemize}

\subsection{Background}

One initiative in TEE standardization has been undertaken by GlobalPlatform (GP), which is a cross industry, non-profit association which identifies, develops and publishes specifications that promote the secure and interoperable deployment and management of multiple applications on secure chip technology. These standardization efforts in GlobalPlatform could resolve the issue of inter-operable TEEs. However, they do not remove the obstacle in gaining access to the requisite hardware nor does simplify the task of developing and testing TAs.

\subsection{OPen-TEE}

The motivations are \begin{itemize}
\item Enable developer access to TEE functionality
\item Provide a fast and efficient prototyping environment
\item Promote research into TEE Services
\item Promote community involvement
\end{itemize}

The requirements are \begin{itemize}
\item Compliance: Our framework should comply with GP’s main interfaces, the Client and Core APIs.
\item Hardware-independence: As a software based solution the framework should not be dependent on a particular TEE hardware environment, neither should the development system itself.
\item Reasonable Performance: The framework must minimize the on-disk footprint and the memory consumption required to run it, the start-up and restart times should also be reasonable.
\item Ease-of-use: The solution should be easily deployed and configured.
\end{itemize}

\paragraph{Architecture}

The architecture is split up in the following parts \begin{itemize}
\item Base: Open-TEE is designed to function as a daemon process in user space. It starts executing Base, a process that encapsulates the TEE functionality as a whole. Base is responsible for loading the configuration and preparing the common parts of the system. Once initialized the Base will fork to create two independent but related processes. One process becomes Manager and the other, Launcher which serves as a prototype for TAs.
\item Manager: Manager can be visualized as Open-TEE’s “operating system”. Its main responsibilities are: managing connections between applications, monitoring TA state, providing secure storage for a TA and controlling shared memory regions for the connected applications.
\item Launcher: The sole purpose of Launcher is to create new TA processes efficiently. When it is first created, Launcher will load a shared library implementing the TEE Core API and will wait for further commands from Manager. Manager will signal Launcher when there is a need to launch a new TA. Upon receiving the signal, Launcher will clone itself. The clone will then load the shared library corresponding to the requested TA.
\item Trusted Application Processes: The architecture of the TA processes is inspired by the multi-process architecture. Each process has been divided into two threads. The first handles Inter-Process Communication (IPC) and the second is the working thread, referred to respectively as the IO and TA Logic threads.
\end{itemize}

\subsection{Evaluation}

By following the GP standard and not emulating any specific TEE hardware, Open-TEE is independent of TEE hardware. TAs developed with Open-TEE can be compiled to any target TEE hardware architecture. We have verified that a non-trivial TA developed using Open-TEE has been succesfully compiled and run on a hardware TEE based on ARM TrustZone.

\end{document}